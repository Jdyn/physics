\documentclass[12pt]{article}

\usepackage[english]{babel}
\usepackage[utf8]{inputenc}
\usepackage{fancyhdr}

\usepackage[margin=1in]{geometry}
\usepackage{pgf}
\usepackage{pgfplots}
\usepackage{siunitx}
\usepackage{tikz}
\usepackage{float}
\usepackage{amsmath}
\usepackage{enumitem}

\usepackage[font=small,labelfont=bf]{caption}
\usepackage{pstricks-add}
\usepackage{pgfplotstable}
\usepackage[nodisplayskipstretch]{setspace}

\usetikzlibrary{scopes}
\usetikzlibrary{angles,quotes}
\usetikzlibrary{calc}
\pgfplotsset{compat=1.5}

\graphicspath{ {/} }

\newcommand*{\I}{\imath}
\newcommand*{\J}{\jmath}
\newcommand{\norm}[1]{\lvert #1 \rvert}

\setlist[enumerate, 1]{label=\alph*.}

\begin{document}
\sisetup{per-mode=symbol}

\begin{filecontents}{data1.csv}
    X	 Y
    20	2.572
    25	1.123
    30	0.64
    34	0.42
    40	0.277
    50	0.125
    60	0.053
    70	0.017
    };
\end{filecontents}

\begin{filecontents}{data2.csv}
     X	 Y
    20	0.714
    25	0.555
    30	0.363
    34	0.255
    40	0.212
    50	0.125
    60	0.097
    70	0.053
    };
\end{filecontents}

\begin{titlepage}
    \begin{center}
        \vspace*{1cm}
        \textbf{Magnetic Field of a Bar Magnet}

        \vspace{0.5cm}
        Lab: 07

        \vspace{1cm}

        \textbf{Jaden Moore}

        \vfill

        Orange Coast College\\
        Physics A280L\\
        May 4th, 2021

    \end{center}
\end{titlepage}

\pagestyle{fancy}
\fancyhf{}
\setlength{\headheight}{15pt}
\lhead{Magnetic Field of a Bar Magnet}
\rhead{Lab: 07}
\cfoot{\thepage}

\section{Introduction}
In this lab, we measure the strength of the magnetic field generated by a bar magnet. We then consider the strength of the magnetic field at various distances away from the center of the bar magnet. We predict that as distance from the center of the magnet increases, the strength of the magnetic field decreases.

\section{Magnetic Fields}
Consider a meter stick oriented from east to west on a sheet of paper. First, we position a bar magnet such that its magnetic field points perpendicularly to Earth's magnetic field on the meter stick. We then position a small compass at various distances away from the bar magnet and measure the magnetic field. We accomplish this by measuring the angle $\theta$ that the Earth's magnetic field vector makes with the net vector sum of the Earth's and the bar magnets magnetic field.

We can then use the following formula to calculate the magnetic field:

\[B = 0.5tan(\theta)\]

Below is a table representing the magnetic field of the bar magnet at different distances away.

\newcolumntype{P}[1]{>{\centering\arraybackslash}p{#1}}

\setlength{\tabcolsep}{3pt}
\renewcommand{\arraystretch}{1.25}

\begin{figure}[H]
    \begin{center}
        \begin{tabular}{ P{4cm} P{4cm} P{4cm} }
            \hline
            \multicolumn{3}{c}{Table 1: Magnetic field at various distances} \\
            \hline
            Distance d [cm] & $\theta$ [$^\circ$] & $B$ [G]                  \\
            \hline
            20              & 79                  & 2.572                    \\
            25              & 66                  & 1.123                    \\
            30              & 52                  & 0.640                    \\
            34              & 40                  & 0.420                    \\
            40              & 29                  & 0.277                    \\
            50              & 14                  & 0.125                    \\
            60              & 6                   & 0.053                    \\
            70              & 2                   & 0.017                    \\


            \hline
        \end{tabular}
    \end{center}
\end{figure}

A preliminary analysis of the data indicates that as the distance from the bar magnet increases, the strength of the magnetic field decreases. From this, we can apply a power regression to determine the of best fit for the data. Below is a graph of the acquired from Table 1.

\begin{figure}[H]
    \centering

    \caption[10pt]{Magnetic field over distance}

    \begin{tikzpicture}
        \pgfplotsset{width=11cm,
            legend style={font=\footnotesize}}
        \begin{axis}[
                xlabel={d $[cm]$},
                xmin=0,
                xmax=80,
                ylabel={B $[G]$},
                ymin=0,
                ymax=4,
                legend cell align = left,
                legend pos = north east,
            ]
            \addplot[only marks] table[x=X,y=Y]{data1.csv};
            \addplot[black,smooth, domain=20:80] {225590*x^-3.756};
            \addlegendimage{only marks}
            \addlegendentry{data point}
            \addlegendimage{no markers, red}
            \addlegendentry{$y=225590x^{-3.756}$}
        \end{axis}
    \end{tikzpicture}
\end{figure}

From this, we get the equation of the magnetic field produced by the bar magnet to be:

\begin{equation*}
    \begin{split}
        B = \frac{225590}{d^{3.756}}
    \end{split}
\end{equation*}

Where B is the magnetic field in Gauss, and d is the distance in cm. We can see that the magnetic field falls off at about one over the distance to the fourth power.

\bigskip

Now consider the meter stick positioned such that it is orientated from north to south along the sheet of paper and the bar magnet still orientated from east to west. The table below represents the data collected.

\setlength{\tabcolsep}{2pt}
\renewcommand{\arraystretch}{1.25}

\begin{figure}[H]
    \begin{center}
        \begin{tabular}{ P{4cm} P{4cm} P{4cm} }
            \hline
            \multicolumn{3}{c}{Table 1: Magnetic field at various distances} \\
            \hline
            d [cm] & $\theta$ [$^\circ$] & $B$ [G]                           \\
            \hline
            20     & 55                  & 0.714                             \\
            25     & 48                  & 0.555                             \\
            30     & 36                  & 0.363                             \\
            34     & 27                  & 0.255                             \\
            40     & 23                  & 0.212                             \\
            50     & 14                  & 0.125                             \\
            60     & 11                  & 0.097                             \\
            70     & 6                   & 0.053                             \\
            \hline
        \end{tabular}
    \end{center}
\end{figure}

In this case, we notice that the magnetic field of the bar magnet continues to decrease as distance away increases however, we notice that the values of the magnetic field does not decrease as quickly as it did in with the previous orientation.

\begin{figure}[H]
    \centering

    \caption[10pt]{Magnetic field over distance}

    \begin{tikzpicture}
        \pgfplotsset{width=11cm,
            legend style={font=\footnotesize}}
        \begin{axis}[
                xlabel={d $[cm]$},
                xmin=0,
                xmax=80,
                ylabel={B $[G]$},
                ymin=0,
                ymax=1,
                legend cell align = left,
                legend pos = north east,
            ]
            \addplot[only marks] table[x=X,y=Y]{data2.csv};
            \addplot[black,smooth, domain=20:80] {357.27*x^-2.037};
            \addlegendimage{only marks}
            \addlegendentry{data point}
            \addlegendimage{no markers, red}
            \addlegendentry{$y=357.27x^{-2.037}$}
        \end{axis}
    \end{tikzpicture}
\end{figure}

From this, we get the equation of the line of best fit to be:

\begin{equation*}
    \begin{split}
        B = \frac{357.27}{d^{2.037}}
    \end{split}
\end{equation*}

In this case, the magnetic field falls off very close to the inverse square of the distance from the center of the bar magnet. It appears that depending on the orientation of the bar magnet with respect to Earth's magnetic field, the magnetic field of the bar magnet changes.

\section{Conclusion}
From this lab, we get that as the distance away from a magnetic field increases, the strength of the magnetic field decreases quite exponentially. Furthermore, depending on the orientation of a bar magnet relative to the Earth's magnetic field, we get varying results with respect to the strength of the bar magnet's electric field. The greatest obstacle was analyzing the data of the magnetic field at various distances. The biggest takeaway was understanding how the Earth's magnetic field has a considerable impact on the magnetic field of ordinary magnets.
\end{document}