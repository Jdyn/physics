\documentclass[12pt]{article}

\usepackage[english]{babel}
\usepackage[utf8]{inputenc}
\usepackage{fancyhdr}

\usepackage[margin=1in]{geometry}
\usepackage{pgf}
\usepackage{pgfplots}
\usepackage{siunitx}
\usepackage{tikz}
\usepackage{float}
\usepackage{amsmath}
\usepackage{enumitem}

\usepackage[font=small,labelfont=bf]{caption}
\usepackage{pstricks-add}
\usepackage{pgfplotstable}
\usepackage[nodisplayskipstretch]{setspace}

\usetikzlibrary{scopes}
\usetikzlibrary{angles,quotes}
\usetikzlibrary{calc}
\pgfplotsset{compat=1.5}
\graphicspath{ {/} }

\newcommand*{\I}{\imath}
\newcommand*{\J}{\jmath}
\newcommand{\norm}[1]{\lvert #1 \rvert}

\setlist[enumerate, 1]{label=\alph*.}

\begin{document}
\sisetup{per-mode=symbol}

\begin{titlepage}
    \begin{center}
        \vspace*{1cm}
        \textbf{Mass Spectrometer}

        \vspace{0.5cm}
        Lab: 08

        \vspace{1cm}

        \textbf{Jaden Moore}

        \vfill

        Orange Coast College\\
        Physics A280L\\
        May 12th, 2021

    \end{center}
\end{titlepage}

\pagestyle{fancy}
\fancyhf{}
\setlength{\headheight}{15pt}
\lhead{Mass Spectrometer}
\rhead{Lab: 08}
\cfoot{\thepage}

\section{Introduction}
Consider a negatively charged particle that enters a region with a constant uniform magnetic field directed inward, and a constant uniform electric field directed downward. If the particle is able to pass through the region with crossed electric and magnetic fields, then the particle is considered ``selected'', at which point it then enters a region where only the magnetic field is present.

\section{Data and Analysis}
Consider the experiment provided by Physlet\textregistered \space Physics - Exploration 27.3: Mass Spectrometer. Below we analyze the properties of the experiment and obtain experimental and theoretical values for the experiment.

\bigskip

(a) First, let the initial velocity $V$ is 50 m/s, the magnetic field $B$ is 0.5 T, the mass 0.3 g, and the electric charge $q$ be $\SI{-1}{x 10^{-3} C}$. We can predict that the electric field is 25 N/C because it is directly proportional to the velocity times the magnetic field.

\[E_{exp} = \SI{25}{V/m}\]
\[E_{theory} = VB = \SI{25}{V/m}\]
\[\text{\% error} = 0 \% \]

(b) If you only change the magnetic field, the particle would no longer be ``selected'' because the particle can only be selected if $E=BV$ which does not hold when $B$ changes.

\bigskip

(c) However, if you change the mass of the charge, the particle would still be ``selected'' because  the mass is directly proportional to the radius. As the mass changes, the only value affected is the radius.

\bigskip

(d) After the particle enters the region where only the electric field is present, the only force acting on the particle is the electric field directed downward, which causes the particle to curve.

Experimentally, we select both ends of the path to determine the diameter of the semi-circular path created by the charged particle. In this case we get that $P_1 = (-2.5, 7.6)$ and $P_2 = (-2.5, -12.4)$. This leaves us with a diameter of 20 m, and a radius of 10 m.

Similarly, from theory we get that


\begin{equation*}
    \begin{split}
        r &= \frac{mv}{qB} \\
        r &= \frac{(\SI{0.1}{x10^{-3}})(50)}{10^{-3}(0.5)} \\
        r &= 10 m \\
    \end{split}
\end{equation*}

From this, we get that the percent error is

\[\% error = 0\%\]

\bigskip

(e) (f) We predict that the mass of the particle which produces the curved path with radius 10 m to be $10^{-4}$ kg.
Below we derive the theoretical equation for the mass and compare it to the experimental value obtained in the experiment.

\begin{equation*}
    \begin{split}
        m &= \frac{qBr}{v} \\
        m &= \frac{qB^2r}{E} \\
        m &= \frac{(10^{-3})(0.5)^2(10)}{25} \\
        m &= 10^{-4} kg \\
    \end{split}
\end{equation*}

From this, we get that the percent error of the mass of the charged particle is

\[\% error = 0\%\]

\bigskip

(g) We can then measure the charge-to-mass ratio $\frac{q}{m}$ to be:

\[\frac{q}{m}_{measured} = \frac{E}{B^2r} = \frac{25}{(0.5)^2(10)} = \SI{10}{C/kg}\]

\section{Conclusion}
From this lab, we get that a negatively charged particle that enters a region with a constant uniform magnetic field directed inward, and a constant uniform electric field directed downward can be directed toward an opening based on the magnetic and electric fields. Scientist can then calculate the mass of the particle using this technique. The greatest obstacle was analyzing the motion of the particle and determining the radius of its motion after entering the electric field. The biggest takeaway was gaining a better understanding of how scientist measure the mass of small particles using phyiscs.
\end{document}