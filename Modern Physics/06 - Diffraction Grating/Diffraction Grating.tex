\documentclass[12pt]{article}

\usepackage[english]{babel}
\usepackage[utf8]{inputenc}
\usepackage{fancyhdr}

\usepackage[margin=1in]{geometry}
\usepackage{pgf}
\usepackage{pgfplots}
\usepackage{siunitx}
\usepackage{tikz}
\usepackage{float}
\usepackage{amsmath}
\usepackage{enumitem}
\usepackage{textcomp}

\usepackage[font=small,labelfont=bf]{caption}

\usetikzlibrary{scopes}
\usetikzlibrary{angles,quotes}
\usetikzlibrary{calc}
\graphicspath{ {/} }
\pgfplotsset{compat=1.5}

\newcommand*{\I}{\imath}
\newcommand*{\J}{\jmath}
\newcommand{\norm}[1]{\lvert #1 \rvert}

\setlist[enumerate, 1]{label=\alph*.}

\begin{document}
\sisetup{per-mode=symbol}

\begin{titlepage}
    \begin{center}
        \vspace*{1cm}
        \textbf{Diffraction Grating}

        \vspace{0.5cm}
        Lab: 06

        \vspace{1cm}

        \textbf{Jaden Moore}

        \vfill

        Orange Coast College\\
        Physics A285L\\
        November 14th, 2021

    \end{center}
\end{titlepage}

\pagestyle{fancy}
\fancyhf{}
\setlength{\headheight}{15pt}
\lhead{Diffraction Grating}
\rhead{Lab: 06}
\cfoot{\thepage}

\section{Introduction}
In this lab, we measure the distance between the slits on a diffraction grating glass slide by analyzing the diffraction pattern projected onto a screen and compare it to the theoretical value. We then experimentally measure the distance between bright spots and the central bright maxima and compare it to the theoretical values.

\section{The Spacing for a Diffraction Grating}
Consider the experiment provided by Physlet\textregistered \space Physics Exploration 38.2: Diffraction Grating. The location of the intensity maxima on a screen created by any diffraction grating is calculated such that

\begin{equation}
    \begin{split}
        dsin(\theta) &= m \lambda_{light}
    \end{split}
\end{equation}
Where $d$ is the distance between adjacent slits. We set the experiment up such that the distance between slits is $d_{theory}=\SI{3663}{nm}$. That is, there are 273 slits per millimeter. Thus, we can experimentally calculate the distance between slits such that

\begin{equation}
    \begin{split}
        d &= \frac{m \lambda_{light}}{sin(\theta)}
    \end{split}
\end{equation}

Where m = 1,2. Below we put into a table the experimental data gathered from different wavelengths of light.

\newcolumntype{P}[1]{>{\centering\arraybackslash}p{#1}}
\setlength{\tabcolsep}{4pt}
\renewcommand{\arraystretch}{1.2}

\begin{figure}[H]
    \begin{center}
        \begin{tabular}{ P{2cm}| P{2cm} P{2cm} P{2cm} P{2cm} P{2cm} }
            \hline
            \multicolumn{6}{c}{Table 1: Diffraction Pattern Data}                                                                     \\

            \hline
            $\lambda$ [nm] & $\theta_{m=1}$ [$^\circ$] & $\theta_{m=2}$ [$^\circ$] & $d_{m=1}$ [nm] & $d_{m=2}$ [nm] & $d_{avg}$ [nm] \\
            \hline
            589            & 9.2                       & 18.4                      & 3683.99        & 3655.37        & 3669.68        \\
            505            & 8.0                       & 16.0                      & 3628.58        & 3664.24        & 3646.41        \\
            470            & 7.3                       & 15.0                      & 3649.19        & 3631.88        & 3640.54        \\
            \hline
        \end{tabular}
    \end{center}
\end{figure}

Thus, we can calculate $d_{exp}$ by taking the average of the $d_{avg}$ values for each wavelength in Table 1. That is,

\begin{equation}
    \begin{split}
        d_{exp} &= \frac{3669.68 + 3646.41 + 3640.54}{3} = \SI{3652.21}{nm}
    \end{split}
\end{equation}

We can then calculate the percent error between the theoretical distance $d_{theory}$ and the experimental value such that

\begin{equation}
    \begin{split}
        \text{\% error} &= \left( \frac{d_{exp} - d_{theory}}{d_{theory}} \right) 100 = \left( \frac{3652.21nm - 3663nm}{3663nm} \right) 100 = \SI{0.295}{\percent} \\
    \end{split}
\end{equation}

In this case, the percent error indicates that the experimental value is within 1\% of the theoretical value which shows that the experiment is relatively accurate.

\section{Measuring the Location of the Diffraction Maxima}
From the experiment, we can calculate the distance between the central bright maxima and any other diffraction maxima that corresponds to the $mth$ bright fringe. If we let $x$ be the distance from the middle grating to the central bright maximum, we can calculate the distance such that

\begin{equation}
    \begin{split}
        y_m &= xtan(\theta_m) \\
    \end{split}
\end{equation}

Using the previous case of yellow light where $\lambda = 589$nm, and taking the distance between the grating and the screen $x=500$cm. We can compare the theoretical distances $y_m$ to the experimental distances to obtain a percent error between the two. We can calculate the experimental values by simply measuring the distance between the central bright maxima and the $mth$ maxima in the experiment. Below we into a table the locations of the diffraction maxima for yellow light.

\setlength{\tabcolsep}{4pt}
\renewcommand{\arraystretch}{1.2}

\begin{figure}[H]
    \begin{center}
        \begin{tabular}{ P{2cm} P{2cm} P{2cm} P{2cm} P{2cm} P{2cm} }
            \hline
            \multicolumn{6}{c}{Table 2: Location of Diffraction Maxima for Yellow Light (589nm) }               \\

            \hline
            $x$ [cm] & m  & $\theta_m$ [$^\circ$] & $y_{m-theory}$ [cm] & $y_{m-exp}$ [cm] & $\%$ error \\
            \hline
            500      & 2  & 18.4                  & 170.21              & 171              & 0.462      \\
            500      & 1  & 9.2                   & 80.98               & 81               & 0.022      \\
            500      & -1 & 9.2                   & 80.98               & 81               & 0.022      \\
            500      & -2 & 18.4                  & 170.21              & 170              & 0.125      \\
            \hline
        \end{tabular}
    \end{center}
\end{figure}
(1) In this case, we get that the percent error was within 1\% of the theoretical values. (2) Additionally, if the frequency of the light is decreased, the position of the maxima relative to the central axis decreases.
\section{Conclusion}
In this lab, we get that the distance between adjacent slits on a glass grating can be calculated by analyzing the diffraction pattern projected onto a screen by the grating slide. Using this method, we are able to calculate the distance within 1\% of the theoretical value, which indicates the experiment was a success. Furthermore, the pattern projected onto the screen was consistent with what theory predicted the pattern to be. That is, the distance between the central bright and any $mth$ diffraction maxima was equivalent to $y_m = xtan(\theta_m)$ within 0.5\% error. The biggest takeaway was gaining a better understanding of how the theory can be applied to and is consistent with real world phenomena.
\end{document}