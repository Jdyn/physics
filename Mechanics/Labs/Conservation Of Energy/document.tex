\documentclass[12pt]{article}

\usepackage[english]{babel}
\usepackage[utf8]{inputenc}
\usepackage{fancyhdr}

\usepackage{geometry}
\usepackage{pgf}
\usepackage{pgfplots}
\usepackage{siunitx}
\usepackage{tikz}
\usepackage{float}
\usepackage{amsmath}

\usetikzlibrary{scopes}
\usetikzlibrary{angles,quotes}
\usetikzlibrary{calc}
\pgfplotsset{compat=1.5}

\begin{document}
\sisetup{per-mode=symbol}

\begin{titlepage}
    \begin{center}
        \vspace*{1cm}
        \textbf{Conservation Of Energy}

        \vspace{0.5cm}
        Lab: 08

        \vspace{1cm}

        \textbf{Jaden Moore}

        \vfill

        Orange Coast College\\
        Physics A185L\\
        November 1st, 2020

    \end{center}
\end{titlepage}

\pagestyle{fancy}
\fancyhf{}
\setlength{\headheight}{15pt}
\lhead{Conservation Of Energy}
\rhead{Lab: 08}
\cfoot{\thepage}

\section{Introduction}
In this lab, we discuss whether the total mechanical energy of a system remains constant throughout its motion. Consider a spring hanging in a fixed position with a mass attached to the hanging end of the spring. We let the spring and hanging mass oscillate in uniform and analyze its motion over time to determine whether the total energy in the system is conserved. We use a motion sensor positioned below the system to measure its position and velocity over time. In this experiment we consider the spring to be massless.
\section{Position And Velocity}
Consider a mass oscillating with a hanging spring in uniform. From Figure 2, we notice that halfway through the downward motion of the system, the velocity switches from decreasing to increasing. More closely, as the position hits a maximum or minimum, the curvature of the velocity graph switches from concave down to concave up. From this, we conclude that the maxima and minima of the position graph is the inflection points of the velocity graph. And at the inflection points of the velocity graph, the velocity is zero. Thus, we gather that as motion of the system reaches a maximum or minimum the motion of the system slows down.
\section{Energy Of The System}
Throughout the motion of the system, Figure 3 suggests that the energy in the system switches between the mass and the spring depending on the point in time of oscillation. As the gravitational potential energy increases, the elastic potential energy of the spring decreases. That is, as the spring decompresses, the energy in the system is effectively transferred from the spring to the mass. The elastic potential energy is converted into gravitational potential energy. If we relate this to the position graph, this occurs when the system is moving downward toward the negative y-direction. That is, as the motion reaches $Y_\text{max}$, gravitational potential energy is at a maximum. However, as the energy is converted between the mass and the spring, the total energy in the system remains constant, which indicates that the energy is conserved throughout the motion. That is, as the spring decompresses, the energy is transferred from the spring, into the mass. This indicates that gravity and the force exerted by the spring are conservative. Furthermore, an analysis of the kinetic energy in the system indicates that the maxima and minima of the kinetic energy is the inflection points of the gravitational and elastic potential energy graphs. The kinetic energy is very near to zero, but appears to oscillate with the motion of the system. Since the kinetic energy is not constant, we can conclude that the acceleration and velocity of the system is changing.
\section{Conservation Of Mechanical Energy}
Conservation of mechanical energy states that the total energy in a system is constant if all the forces acting in the system are conservative. That is, the sum of potential and kinetic is constant. In this experiment, we should expect the total energy in the system to remain constant because gravitational and elastic potential energy are conservative forces. From figure 3, we saw that the total energy in the system remained constant throughout the motion of the system. This proves that the forces acting in the system are conservative.
\section{Conclusion}
In closing, We expected correctly that the total energy in the system would remain constant as the only forces acting on the system are conservative. Throughout the motion of the system, the energy within the system is converted between elastic and gravitational potential energy depending on the point in time of its oscillation. However, during the conversion of energy in each oscillation, no energy was lost which indicates that the forces acting in the system are conservative.
\end{document}
