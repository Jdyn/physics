\documentclass[12pt]{article}

\usepackage[english]{babel}
\usepackage[utf8x]{inputenc}
\usepackage{fancyhdr}

\usepackage{geometry}
\usepackage{siunitx}
\usepackage{amssymb}
\usepackage{amsmath}

\title{
    Chapter 1 \\
    Units, Physical Quantities, and Vectors
    }
\author{Jaden Moore}
\date{October 29, 2020}
\linespread{1.5}

\begin{document}
\sisetup{per-mode=symbol}

\maketitle

\pagestyle{fancy}
\fancyhf{}
\setlength{\headheight}{15pt}
\lhead{Units, Physical Quantities, and Vectors}
\rhead{Chapter: 01}
\cfoot{\thepage}

\section{Standards and Units}
The metric system is also known as the “SI” system
of units. (SI ≡ Système International).

\begin{itemize}
    \item[A.] \textbf{\underline{Length}} — The unit of length in the metric system is the meter.
          A meter is the distance that light travels in a vacuum.
          ($\frac{1}{299,792,457}$ seconds).

    \item[B.] \textbf{\underline{Mass}} — The unit of \textit{mass} in the metric system is the kilogram.
          A \textit{\underline{kilogram}} is the mass of a particular cylinder of
          platinum-iridium alloy kept at the International Bureau of
          Weights and Measures in Paris, France.

    \item[C.] \textbf{\underline{Time}} — The unit of \textit{time} in the metric system is the \textit{\underline{second}}. A
          second is the time required for 9,192,631,770 cycles of a particular microwave radiation associated with cesium
          atoms that occurs in the transition between its two lowest
          energy states.
\end{itemize}

\section{Uncertainty and Significant Figures}
When a value is not known precisely, the amount of
uncertainty is usually called an “error”. Error represents
uncertainty and has nothing to do with mistakes or
sloppiness.

A significant figure is a reliably known digit. When we
say that a quantity has the value 3, we mean by
convention that the value could actually be anywhere
between 2.5 and 3.5. However, if we say that the value is
3.0, then we mean the value lies between 2.95 and 3.05.
\subsection{Significant Figures in Addition or Subtraction:}
The number of decimal places in the result should
equal the smallest number of decimal places of any term from the original equation.

\begin{equation} \label{eq1}
    23.45 + 1.345 = 24.795 ⇒ 24.80
\end{equation}

\begin{equation} \label{eq2}
    56 - 34.56 = 21.44 ⇒ 21
\end{equation}

\subsection{Significant Figures in Multiplication or Division:}
The number of significant figures in the final product
is the same as the number of significant figures in the
factor with the lowest number of significant figures.

\begin{equation} \label{eq3}
    123.56(7.89) = 974.8884 ⇒ 975
\end{equation}

\begin{equation} \label{eq4}
    \frac{564}{0.0034} = 165882.352941 ⇒ 1.7x10^5
\end{equation}

\section{Vectors and Vector Addition}
\begin{itemize}
    \item[A.] \textbf{\underline{Scalar}} — A \textit{scalar} quantity is a physical quantity that has only magnitude. Examples:
          \begin{enumerate}
              \item Mass m (in kilograms, kg)
              \item time t (in seconds, s or sec)
              \item temperature T (in Kelvin, K)
              \item volume V (in cubic meters, $m^3$)
              \item density $\rho$ (in $kg/m^3$)
              \item energy \textit{E} (in Joules, J)
              \item distance \textit{d} (in meters, m)
              \item speed v (in meters per second, m/s)
              \item electric charge q (in Coulombs, C)
          \end{enumerate}

    \item[B.] \textbf{\underline{Vector}} — A \textit{vector} quantity is a physical quantity that has both magnitude and a direction. Examples:
          \begin{enumerate}
              \item displacement $\Delta r$ (in meters, $m$)
              \item velocity $\vec{v}$ (in meters per second, $m/s$)
              \item acceleration $\vec{a}$ (in meters per second square, $m/s^2$)
              \item force $\vec{F}$ (in Newtons, $N$)
              \item linear momentum $\vec{p}$ (in $kg m/s$)
              \item angular momentum $\vec{L}$(in $kg m^2 / sec$)
          \end{enumerate}
\end{itemize}

\paragraph{}

\subsection{The Displacement Vector $\Delta r$}
The displacement vector $\Delta r$ of an object is defined as the
vector whose magnitude is the shortest distance between
the initial and final positions of the object, and whose
direction points from the initial position to the final
position.

\subsection{Components of Vectors}

\end{document}
